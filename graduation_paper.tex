% !TEX program = uplatex
% (環境に合わせて platex/uplatex/lualatex 等に変更してください)
\documentclass[a4paper,12pt]{jsbook}

% --- packages ---
\usepackage{amsmath,amssymb,amsthm}
\usepackage{mathtools}
\usepackage{geometry}
\usepackage{hyperref}
\usepackage{bookmark}

\geometry{margin=25mm}
\hypersetup{
  colorlinks=true,
  linkcolor=blue,
  urlcolor=blue,
  citecolor=blue
}

% --- theorem environments (必要に応じて増やす) ---
\newtheorem{theorem}{定理}[chapter]
\newtheorem{lemma}[theorem]{補題}
\newtheorem{proposition}[theorem]{命題}
\newtheorem{corollary}[theorem]{系}
\theoremstyle{definition}
\newtheorem{definition}[theorem]{定義}
\theoremstyle{remark}
\newtheorem{remark}[theorem]{注意}

% --- metadata ---
\title{チェバの定理の高次元一般化と Lean/mathlib への形式化}
\author{(氏名)}
\date{(提出年月)}

\begin{document}
\maketitle

\frontmatter
\chapter*{要旨}
\addcontentsline{toc}{chapter}{要旨}
% ここに要旨(目的・主結果・実装成果)を記述

\tableofcontents

\mainmatter

% =========================
\chapter{序論}
\section{研究背景と動機}
% (すでに一部 tex 中に書いてある背景をここに統合)

\section{本研究の目的}
% 例:チェバの定理の一般化を選定し、数学的証明と Lean 形式化を与え、
% mathlib への追加(もしくは追加可能性)を議論する

\section{本研究の貢献}
% 数学面/ライブラリ設計面/形式化面

\section{本論文の構成}
% 各章の概要

% =========================
\chapter{チェバの定理(2次元)とアフィン性}
\section{古典的チェバの定理の定式化}
% 定理のステートメント(どの表現を採用するか明確化:有向比、比、重心座標など)

\section{チェバの定理がアフィンな結果であること}
% 内積・距離に依存しないこと、アフィン変換で不変な構造に基づくことなど

\section{形式化で必要となる構造の整理}
% Lean で扱う型・構造(AffineSpace, Line, Collinear, ratio 等)を棚卸し

% =========================
\chapter{高次元への一般化の候補}
\section{一般化の設計空間}
% 「何を固定し、何を一般化するか」の設計観点を列挙

\section{候補A:平面の三角形を保ち周囲空間のみ高次元化}
% 埋め込み一般化:2次元の命題を n 次元アフィン空間内の 2-平面に還元

\section{候補B:単体(simplex)版のチェバ}
% n-単体での cevian/concurrency 条件(行列式・バリセントリック等)

\section{候補C:射影幾何・行列式・重心座標など別定式化}
% 実装上の都合も含めて整理

\section{比較(数学的自然さ/形式化コスト/再利用性)}
% 表を入れると強い

% =========================
\chapter{mathlib に加える一般化の選定}
\section{評価軸}
% 依存最小化、既存 API との整合、拡張性、証明の見通し、将来の定理への波及

\section{候補の比較}
% 表・箇条書きでよい

\section{採用する一般化と最終ステートメント}
% 数学的な定理文+Lean 的な型・仮定を言語化

% =========================
\chapter{採用した一般化チェバの定理の数学的証明}
\section{設定と定義}
% 点・直線・比(ratio)・cevian の定義・共点性 predicate 等

\section{補題群}
% 比の性質、アフィン結合、重心座標、非退化条件の扱いなど

\section{主定理と証明}
\begin{theorem}[(主定理名)]
% ここに主定理のステートメント
\end{theorem}
\begin{proof}
% ここに証明
\end{proof}

\section{一般化になっている点の明確化}
% 2次元との対応、仮定の強弱、構造の最小性など

% =========================
\chapter{Lean による形式化}
\section{Lean と mathlib の関連基盤}
% import、使用する主要構造(AffineSpace 等)、既存定理との接続

\section{定義の Lean 化}
% 数学の定義 → Lean の定義(型・命名・引数設計)

\section{証明の Lean 化}
% 数学の補題列 → Lean の lemma 群
% tactic / term-style / simp 方針など

\section{実装上の典型的障害と対処}
% coercions, simp 正規形、非退化条件、局所座標化、等

% =========================
\chapter{既存モジュールの転用可能性}
\section{転用できる部分}
% 既存 API の再利用箇所を列挙

\section{転用できない(不足している)部分}
% 欠けている定義・補題・抽象化

\section{依存関係と設計上の制約}
% import の増減、循環依存回避、将来拡張

% =========================
\chapter{追加すべきライブラリと実装上の工夫}
\section{追加ライブラリ一覧(ファイル/モジュール単位)}
% 例:Geometry/Affine/Ceva.lean 等(仮)

\section{各モジュールの設計方針}
% API、命名、simp lemma、非退化条件の表現

\section{実装の工夫事項・特記事項}
% PR を通す観点(一般性・粒度・再利用性・ドキュメント)

\section{使用例・テスト}
% 簡単な例題や rewrite 例

% =========================
\chapter{本ライブラリで示せる別定理と発展例}
\section{近縁定理への接続}
% Menelaus、Desargues、Pascal など(扱える範囲で)

\section{定理群のテンプレ化}
% 共点・共線の抽象化、座標化、barycentric 連携

\section{高次元へのロードマップ}
% simplex 版への拡張など

% =========================
\chapter{結論と今後の展望}
\section{結論}
% 成果の要約(数学/形式化/mathlib 追加)

\section{今後の課題}
% 自動化、API 洗練、他定理拡張、教育教材整備など

\section{Lean 普及と形式化研究の展望}
% あなたの長期ビジョン(自動証明支援)もここで回収

% =========================
\appendix
\chapter{付録:主要 Lean コード抜粋}
% 定義・主定理・重要補題を最小限に抜粋

\chapter{付録:概念対応表(数学用語 ↔ Lean の型/定義)}
% 辞書的な表

\chapter{付録:実装ログ(ハマりどころ集)}
% simp、型クラス、分母0回避、等

\backmatter
\chapter*{謝辞}
\addcontentsline{toc}{chapter}{謝辞}
% 必要なら

\chapter*{参考文献}
\addcontentsline{toc}{chapter}{参考文献}
% thebibliography でも BibTeX/Biber でも可
% \begin{thebibliography}{99}
% \bibitem{...} ...
% \end{thebibliography}

\end{document}
